% El primer párrafo de un resumen no resume el trabajo, es la motivació del mismo y correspondería más bien a la introducción.

% Los juegos siempre han sido objeto de estudio de los investigadores en
% muchos campos de la Ciencia: matemáticas, economía, psicología... En el campo
% de la Inteligencia Artificial es común utilizarlos como bancos de pruebas
% de algoritmos o teorías para solucionar problemas complejos, debido a su
% sencilla especificación, pero complejo modelado.

% Eso hace que este resumen no sea suficiente. Usas términos muy genéricos como "Big Data". No mencionas la dimensión del problema, las decisiones que se han tomado

% En este trabajo se modelará el resultado de partidas del conocido juego de
% estrategia en tiempo real \emph{StarCraft}, utilizando aprendizaje supervisado
% con herramientas para trabajar en entornos de \emph{Big Data}, lo que nos
% permitirá trabajar con la cantidad de datos que se dispone sin problema.

% Una vez más, es un resumen. Este párrafo es tan genérico que se podría aplicar a una cantidad de trabajos considerable. "Más potentes" tampoco es previso, hay que decir qué conclusiones se han sacado sobre los predictores. ¿Son más potentes antes o después de aplicarlos? Recuerda "No Free Lunch"

% Se demostrará con este estudio que, con los datos y herramientas adecuadas,
% es posible predecir el resultado de estas partidas, además de explicar por qué
% utilizando los predictores más potentes para cada modelo.

En este trabajo se ha propuesto modelar el resultado de partidas del conocido
juego de estrategia en tiempo real \emph{StarCraft} utilizando aprendizaje
supervisado, de forma que se pueda predecir el ganador de éstas usando
solamente datos generados por los propios jugadores mientras compiten.

Para conseguir esto se propone realizar un proceso de extracción del
conocimiento o \emph{KDD} sobre una base de datos relacional lo suficientemente
grande: contiene más de 4500 partidas, cada una con decenas de posibles
características a utilizar. Se seleccionarán variables interesantes y, a la vez,
fáciles de entender y utilizar, para realizar predicciones que sean tanto
precisas como interpretables. Sólo así se podrá comprender mejor el desarrollo
de este juego a la vez de conseguir predicciones útiles.

Una vez recogidos y organizados los datos, se les aplicarán técnicas de
aprendizaje que permitan extraer información interesante: tanto el ganador
de las partidas en un momento temprano de éstas como las características que,
a posteriori, dan más información sobre el resultado de éstas. Debido a la
dimensión del problema, las técnicas a utilizar deben estar preparadas para
tratar con grandes cantidades de datos, por lo que se utilizará el ecosistema
de Apache para tal problema, \emph{Spark} junto a la biblioteca de aprendizaje
\emph{MLlib}. Esta elección nos permitirá trabajar con grandes cantidades de
datos sin que los limitados recursos de una computadora personal sea ningún
impedimento.

Como resultado del proyecto obtendremos una serie de modelos que son capaces
de predecir el ganador de una partida con una precisión bastante alta. Además,
algunos de estos modelos ofrecen también un ránking de las variables
más importantes para cada uno. De esta manera se puede evaluar, para cada
modelo, qué variables del problema son las más importantes. Esto nos ofrecerá
distintos modos de ver el ganador de una partida y permitirá completar los
análisis y trabajos que ya hay sobre este complejo videojuego.
