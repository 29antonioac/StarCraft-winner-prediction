% El primer párrafo de un resumen no resume el trabajo, es la motivació del mismo y correspondería más bien a la introducción.
Los juegos siempre han sido objeto de estudio de los investigadores en
muchos campos de la Ciencia: matemáticas, economía, psicología... En el campo
de la Inteligencia Artificial es común utilizarlos como bancos de pruebas
de algoritmos o teorías para solucionar problemas complejos, debido a su
sencilla especificación, pero complejo modelado.

% Eso hace que este resumen no sea suficiente. Usas términos muy genéricos como "Big Data". No mencionas la dimensión del problema, las decisiones que se han tomado
En este trabajo se modelará el resultado de partidas del conocido juego de
estrategia en tiempo real \emph{StarCraft}, utilizando aprendizaje supervisado
con herramientas para trabajar en entornos de \emph{Big Data}, lo que nos
permitirá trabajar con la cantidad de datos que se dispone sin problema.

% Una vez más, es un resumen. Este párrafo es tan genérico que se podría aplicar a una cantidad de trabajos considerable. "Más potentes" tampoco es previso, hay que decir qué conclusiones se han sacado sobre los predictores. ¿Son más potentes antes o después de aplicarlos? Recuerda "No Free Lunch"
Se demostrará con este estudio que, con los datos y herramientas adecuadas,
es posible predecir el resultado de estas partidas, además de explicar por qué
utilizando los predictores más potentes para cada modelo. 
