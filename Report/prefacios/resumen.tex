% Una vez más: el primer párrafo de un resumen es la motivación del mismo. ¿Por qué quiero hacer este trabajo o creo que es necesario? - Jj
% Evita juicios de valor: "conocido", "famoso", "mejor", "peor" - JJ
% Estás adelantando la metodología, y usando un término que puede que el lector no conozca. Además, va de suyo: si tienes el resultado deseado, tienes que usar aprendizaje supervisado por narices.

En este trabajo se aborda el problema de la predicción del resultado en
juegos de estrategia en tiempo real, en particular de \emph{StarCraft}, de
forma que se pueda predecir el ganador de una partida sin completarla.
En particular, se pretende minimizar el tiempo necesario para predecir el
resultado final de una partida de manera precisa, con la motivación
de mejorar el tiempo empleado en la creación de agentes autónomos para el juego.

Además, se pretende explicar
por qué un jugador gana frente a otro con una configuración determinada de la
partida, por lo que se buscará que estos modelos sean interpretables.

% No digas "para", di por qué tienes que hacer eso. Ese conocimiento
% está aquí y allí. He tenido que hacer esto para sacarlo - JJ
Hay que realizar un proceso de extracción del
conocimiento o \emph{KDD} sobre una conjunto de datos lo suficientemente
grande, ya que solo así se conseguirán predicciones fiables e información útil
sobre el desarrollo de las partidas. Los datos escogidos forman parte de una
base de datos relacional con más de 4500 partidas, cada una con decenas de posibles
características a utilizar. Se seleccionarán variables que a priori son
interesantes y fáciles de entender, para realizar predicciones que sean tanto
precisas como interpretables.
% Un proceso de aprendizaje tiene varias fases: preparación o
% extracción de los datos, extracción de características. No hables de
% características "interesantes", sino de características que sean
% independientes, expliquen la mayor parte de la varianza o hayan sido
% extraidas con métodos de extracción de características.
% Para obtener precicciones interpretables necesitas usar un modelo
% que no sea una caja negra. No has dicho que sea ese el
% objetivo. El esquema de un resumen es
% 1, Existe este problema que nadie ha resuelto hasta ahora.
% 2. Quiero resolver este problema específico y la solución tiene que
% tener estas características. Esos serán los objetivos del trabajo
% 3. Voy a usar esta metodología que me permitirá hacerlo de forma
% súperguay y de la mejor posible.

Una vez recogidos y organizados los datos, se les aplicarán técnicas de
aprendizaje que permitan extraer información interesante: tanto el ganador
de las partidas en un momento temprano de éstas como las características que,
a posteriori, dan más información sobre el resultado de éstas. Debido a la
dimensión del problema, las técnicas a utilizar deben estar preparadas para
tratar con grandes cantidades de datos, por lo que se utilizará el ecosistema
de Apache para tal problema, \emph{Spark} junto a la biblioteca de aprendizaje
\emph{MLlib}. Esta elección nos permitirá trabajar con grandes cantidades de
datos sin que los limitados recursos de una computadora personal sea ningún
impedimento.

% Elegir algo tiene que ser un resultado, no partir de ahí. Un TFM
% debe decir: "se examinarán diferentes técnicas aplicables en este
% caso para escoger la que tenga las características que cumplan los
% objetivos" - JJ
% Tengo que decantarme por un modelo en particular, o puedo exponer de
% esta forma las ventajas de uno y de otro? - Antonio

Como resultado del proyecto obtendremos una serie de modelos que son capaces
de predecir el ganador de una partida con una precisión bastante
alta. En particular, modelos de tipo \emph{KNN} y de tipo \emph{Gradient Boosting}
consiguen predicciones muy precisas y muy interpretables, respectivamente.
El primero de ellos consigue predicciones útiles en un estado muy temprano
de las partidas, por lo que es muy adecuado para cumplir el objetivo principal:
determinar el ganador sin jugar una partida completa.
Por otro lado, \emph{Gradient Boosting} consigue predicciones menos fiables,
pero es un modelo interpretable. Por esta razón, se pueden ver qué
características explican por qué un jugador gana al otro, cubriendo el
segundo objetivo del trabajo.



% Un resumen no es un teaser. Tienes que decir qué has obtenido, y qué
% has aprendido en el proceso. "Usando la metodología propuesta, hemos
% cumplido los objetivos planteados, probando que A y B son adecuados
% para C" - JJ
% ¿Es suficiente tal y como lo expongo? - Antonio

% Esto es trabajo futuro. Además, estás planteando un objetivo del
% trabajo. Si es un objetivo, al principio del resumen. Si es trabajo
% futuro, no corresponde ponerlo en este resumen"
