In the field of intelligent agents creation to solve complex problems, a
difficult one is the time to get a solution that exceeds initial expectations.
Specifically, in the creation of agents that play games the problem is
playing a whole match to evaluate them. We are going to study how to reduce
the necessary time to know the winner of a match in a real-time strategy game,
\emph{StarCraft}, using supervised learning. The goal is to get models which
can predict the winner without playing the whole match. These models should be
interpretable, because we want to explain why a player beats the other one.

We have to do a complete Knowledge Discovery in Databases (\emph{KDD}) process
over a large dataset, because it is the only way to get precise predictions and
useful information about the match.

Selected data are a piece of a big database with more than 4500 games, with
tens of possible features to use. Previously used features
by researches and more interesting ones will be taken to improve the state of
the art.

Once the data is gathered and organized, we will apply techniques to get
useful information: the winner of the match in an early stage of it and the
predictors that give more information about the winner. Due to the dimension
of the problem, the used techniques must be prepared to deal with large data,
so we will use the Apache echosystem, \emph{Spark} and \emph{MLlib}. This
choice allow us to deal with large datasets in a desktop computer.

As a result of this project we will obtain some models which are able to
predict the winner with high accuracy. In particular, \emph{KNN} gets high
accuracy predictions, and \emph{Gradient Boosting Tree} gets interpretable ones.
It was not possible to get one accurate and interpretable model at the same time,
so the main objective is divided into two ones: obtaining an accurate model and
getting an interpretable one.

\emph{KNN} gets useful predictions in an early stage of the match, so it is
suitable to accomplish the main objective: to determine the winner
without playing the whole match. \emph{Gradient Boosting Tree} gets less
accurate predictions, but offers a ranked set of features that are important
to predict the winner, so the second objective is covered.
