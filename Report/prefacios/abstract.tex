% Game have always been a target of researchers in a lot of Science fields:
% mathematics, economy, psychology... In Artificial Intelligence is very usual
% to use them as a testing bench for algorithms or theories to resolve complex
% problems, due to their simple specification but complex modelling.
%
% In this work the results of a well-known real time strategy videogame will be
% modelled,
% \emph{StarCraft}, using supervised learning with \emph{Big Data} tools, which
% will allow us to work with a very large dataset without problem.
%
% We will prove that with appropiate data and tools it is possible to
% predict the outcome of matches, explaining why using the best predictors for
% each model.

% No sé si está sincronizado con el español, pero tienes que seguir las reglas que he puesto en todos sitios. Motivación, objetivos, metodología, resultados, hilados como una narrativa. Además, modelar es uno de los objetivos. El otro es optimizar el tiempo necesario para obtener el resultado.
In this work it has been proposed to model the matches' outcomes of the very
well-known real time strategy videogame \emph{StarCraft}. Supervised learning
will be used, so the winner of a match could be predicted using automatically
generated data from the players.


To do this it has been proposed to do a complete process of Knowledge Data
Discovering \emph{KDD} on a relational database, which is very large: it
% kdd = KNOWLEDGE DISCOVERY IN DATABASES!!!!! - JJ
% Revisa el inglés, por favor. 
contains more than 4500 matches, every one with a big amount of possible
features to use. It will be selected interesant and understable features to
get good predictions and explainable ones. In this way the game could be
well-known and get useful predictions.

When the data is completely organised, it will apply techniques to extract
useful information: the winner of a match in an early moment and the features
that explain this prediction. Due to the problem's dimension, used techniques
must be ready to deal with a big amount of data, so Apache ecosystem will
be used: \emph{Spark} together with the machine learning library \emph{MLlib}.
This choice will allow an usual desktop computer to deal with all this data
without any problems.

As a result of this proyect we will obtain a set of models which are able to
predict the winner of a match with a big accuracy. Furthermore, some of this
models have a ranking of features by importance. With this ranking, features
could be analyzed for each model, allowing the researcher to understand better
why a player wins, completing the previous analysis and works of this complex
videogame.
