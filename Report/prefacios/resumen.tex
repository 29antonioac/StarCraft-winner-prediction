% Motivación es: Existe este problema en el mundo, que no ha sido resuelto o no a nuestra entera satisfacción. Nosotros creemos que de esta forma se puede solucionar. La motivación es esencial porque indica conocimiento del estado del arte y del problema.

En el ámbito de la creación de agentes inteligentes para resolver problemas,
una de las muchas dificultades que existen es el tiempo que se tarda en conseguir
una solución que supere las expectativas iniciales. Más específicamente,
en la creación de agentes que juegan a determinados juegos, conocidos como
\emph{bots}, también se presenta este problema, ya que necesitan jugar
partidas completas para poder evaluarlos. Por tanto, se va a estudiar cómo
reducir el tiempo necesario para saber el resultado de una partida de un
juego de estrategia en tiempo real, \emph{StarCraft}, utilizando aprendizaje
supervisado. Se buscará un modelo que sea capaz de predecir el ganador sin jugar
toda la partida. Estos modelos deberían ser interpretables, ya que
también se busca explicar por qué un jugador gana a otro.
% En este trabajo se aborda el problema de la predicción del resultado en
% juegos de estrategia en tiempo real, en particular de \emph{StarCraft}, de
% forma que se pueda predecir el ganador de una partida sin completarla.
% En particular, se pretende minimizar el tiempo necesario para predecir el
% resultado final de una partida de manera precisa, con la motivación
% de mejorar el tiempo empleado en la creación de agentes autónomos para el juego. % Tienes que empezar por esta motivación. ¿Por qué este problema es importante? - JJ

% Además, se pretende explicar
% por qué un jugador gana frente a otro con una configuración determinada de la
% partida, por lo que se buscará que estos modelos sean interpretables.
% No uses además. O es objetivo o no lo es. No es "como el Pisuerga pasa por Valladolid, vamos a hacer también esto". ¿Estás interesado en el modelo interpretable desde el principio? ¿Eso resuelve el problema que es la motivación de todo el trabajo? - JJ

% No digas "para", di por qué tienes que hacer eso. Ese conocimiento
% está aquí y allí. He tenido que hacer esto para sacarlo - JJ
Hay que realizar un proceso de extracción del
conocimiento o \emph{KDD} sobre una conjunto de datos lo suficientemente
grande, ya que solo así se conseguirán predicciones fiables e información útil
sobre el desarrollo de las partidas.
% Si hablas de KDD, tendrás que formular parte del problema como un problema KDD. Tendrás que mostrar, en resultados y conclusiones, qué conocimiento has extraído. - JJ
% No sé si te entiendo, ¿es que no he mostrado los resultados correctamente?
Los datos escogidos forman parte de una
base de datos relacional con más de 4500 partidas, cada una con decenas de posibles
características a utilizar. Se seleccionarán variables utilizadas anteriormente
por investigadores y alguna más que a priori parezcan interesantes para mejorar
el estado del arte.
% Un proceso de aprendizaje tiene varias fases: preparación o
% extracción de los datos, extracción de características. No hables de
% características "interesantes", sino de características que sean
% independientes, expliquen la mayor parte de la varianza o hayan sido
% extraidas con métodos de extracción de características.
% Para obtener precicciones interpretables necesitas usar un modelo
% que no sea una caja negra. No has dicho que sea ese el
% objetivo. El esquema de un resumen es
% 1, Existe este problema que nadie ha resuelto hasta ahora.
% 2. Quiero resolver este problema específico y la solución tiene que
% tener estas características. Esos serán los objetivos del trabajo
% 3. Voy a usar esta metodología que me permitirá hacerlo de forma
% súperguay y de la mejor posible.
% O aquí no has hecho nada o no lo has hecho bien. Sigues hablando de "interesantes" y no de variables "que reflejan toda la varianza".
% El estudio inicial de qué variables se cogen se hizo por "vamos a intentar
% mejorar este estudio, vamos a coger sus variables y unas pocas más" - Antonio

Una vez recogidos y organizados los datos, se les aplicarán técnicas de
aprendizaje que permitan extraer información interesante: tanto el ganador
de las partidas en un momento temprano de éstas como las características que,
a posteriori, dan más información sobre el resultado de éstas. Debido a la
dimensión del problema, las técnicas a utilizar deben estar preparadas para
tratar con grandes cantidades de datos, por lo que se utilizará el ecosistema
de Apache para tal problema, \emph{Spark} junto a la biblioteca de aprendizaje
\emph{MLlib}. Esta elección nos permitirá trabajar con grandes cantidades de
datos sin que los limitados recursos de una computadora personal sea ningún
impedimento.

% Elegir algo tiene que ser un resultado, no partir de ahí. Un TFM
% debe decir: "se examinarán diferentes técnicas aplicables en este
% caso para escoger la que tenga las características que cumplan los
% objetivos" - JJ
% Tengo que decantarme por un modelo en particular, o puedo exponer de
% esta forma las ventajas de uno y de otro? - Antonio
% Tendrás que decantarte por el que hayas elegido... - JJ

Como resultado del proyecto obtendremos una serie de modelos que son capaces
de predecir el ganador de una partida con una precisión bastante
alta. En particular, modelos de tipo \emph{KNN} y de tipo \emph{Gradient Boosting}
consiguen predicciones muy precisas y muy interpretables, respectivamente. No
ha sido posible tener un único modelo preciso e interpretable, ya que \emph{KNN}
se basa en la distancia entre observaciones, tomando cada muestra de validación
y viendo cuáles de las de entrenamiento son las más cercanas.
Por tanto, el objetivo principal pasan a ser dos: conseguir un modelo con gran
precisión y por otro lado, uno interpretable.


\emph{KNN} consigue predicciones útiles en un estado muy temprano
de las partidas, por lo que es muy adecuado para cumplir el objetivo principal:
determinar el ganador sin jugar una partida completa.
Por otro lado, \emph{Gradient Boosting} consigue predicciones menos fiables,
pero es un modelo interpretable. Por esta razón, se pueden ver qué
características explican por qué un jugador gana al otro, cubriendo el
segundo objetivo del trabajo.
% Te recuerdo que no hay dos objetivos en este resumen, sino un objetivo y "Además". Deja claro cuál es el objetivo (modelo rápido e interpretable), aunque al final te conformas con dos subobjetivos: modelo rápido o modelo interpretable - JJ


% Un resumen no es un teaser. Tienes que decir qué has obtenido, y qué
% has aprendido en el proceso. "Usando la metodología propuesta, hemos
% cumplido los objetivos planteados, probando que A y B son adecuados
% para C" - JJ
% ¿Es suficiente tal y como lo expongo? - Antonio
% Casi, pero no del todo - JJ
% Esto es trabajo futuro. Además, estás planteando un objetivo del
% trabajo. Si es un objetivo, al principio del resumen. Si es trabajo
% futuro, no corresponde ponerlo en este resumen"
